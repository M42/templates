% Tipo de documento (presentación)
\documentclass[usenames,dvipsnames]{beamer}

% Cargar el tema
\usetheme{metropolis}

% Configuración de LaTeX
\usepackage[spanish]{babel}
\usepackage[utf8]{inputenc}

% Configuración básica del tema
\metroset{
  % tema oscuro ('dark') o claro ('light'). No tiene efecto al usar la
  % paleta de colores más adelante
  background=light,
  % 'none' para eliminar la diapositiva inicial de cada sección
  sectionpage=progressbar,
  % 'progressbar' o 'simple' para añadir una diapositiva inicial a cada subsección
  subsectionpage=none,
  % contador de página: 'none', 'counter' o 'fraction'
  numbering=none,
  % barra de progreso: 'none', 'head', 'frametitle' o 'foot'
  progressbar=frametitle,
  % fondo de los bloques estilo teorema: 'transparent' o 'fill'
  block=fill,
}

% Fuentes
% Es necesario tener instaladas las fuentes que quieras usar. Por
% ejemplo, con el paquete ttf-google-fonts-git de la AUR.
\setsansfont{Lato}
\setmonofont{Source Code Pro}

% Paleta de colores
\definecolor{accent}{RGB}{151, 186, 88}
\colorlet{darkaccent}{accent!70!black}
\definecolor{foreground}{RGB}{0, 0, 0}
\definecolor{background}{RGB}{255, 255, 255}

% Insertar los colores en el tema
\setbeamercolor{normal text}{fg=foreground, bg=background}
\setbeamercolor{alerted text}{fg=darkaccent, bg=background}
\setbeamercolor{example text}{fg=foreground, bg=background}
\setbeamercolor{frametitle}{fg=background, bg=accent}

\setbeamercolor{headtitle}{fg=background!70!accent,bg=accent!90!foreground}
\setbeamercolor{headnav}{fg=background,bg=accent!90!foreground}
\setbeamercolor{section in head/foot}{fg=background,bg=accent}

%
\defbeamertemplate*{headline}{miniframes theme no subsection}{
  % Caja para mostrar título y autor encima de cada diapositiva
  %% \begin{beamercolorbox}[ht=2.5ex,dp=1.125ex,
  %%     leftskip=.3cm,rightskip=.3cm plus1fil]{headtitle}
  %%   {\usebeamerfont{title in head/foot}\insertshorttitle}
  %%   \hfill
  %%   \leavevmode{\usebeamerfont{author in head/foot}\insertshortauthor}
  %% \end{beamercolorbox}
  %% \begin{beamercolorbox}[colsep=1.5pt]{upper separation line head}
  %% \end{beamercolorbox}

  % Caja para mostrar navegación encima de cada diapositiva
  \begin{beamercolorbox}{headnav}
    \vskip2pt\textsc{\insertnavigation{\paperwidth}}\vskip2pt
  \end{beamercolorbox}
  \begin{beamercolorbox}[colsep=1.5pt]{lower separation line head}
  \end{beamercolorbox}
}
